\documentclass[titlepage]{article}

\usepackage{graphicx}
\usepackage{lastpage}
\usepackage{fancyhdr}
\usepackage{hyperref}
\usepackage{amsmath}
\usepackage{tcolorbox}
\usepackage{makecell}
\usepackage{cancel}
\usepackage{geometry}
\usepackage{color}
\usepackage{tikz}
\usetikzlibrary{shapes,arrows}
\usepackage{xcolor}
\usepackage{array,multirow}
\usepackage{booktabs}
\usepackage[framed,numbered,autolinebreaks,useliterate]{mcode}

\newgeometry{left=2cm,bottom=2cm}

\hypersetup{
    colorlinks=true,
    linkcolor=blue, 
    filecolor=magenta,
    urlcolor=blue
}

\setcounter{secnumdepth}{0}

\pagestyle{fancy}
\fancyhf{}
\renewcommand{\headrulewidth}{0pt}
\rfoot{\thepage \hspace{1pt} of \pageref{LastPage}}

\definecolor{mygreen}{RGB}{28,172,0}
\definecolor{mylilas}{RGB}{170,55,241}
\definecolor{com}{rgb}{0,0.6,0.3}

\colorlet{kw}{blue}

\lstset{language=Matlab,
    breaklines=true,
    morekeywords={matlab2tikz},
    keywordstyle=\color{blue},
    morekeywords=[2]{1}, keywordstyle=[2]{\color{black}},
    identifierstyle=\color{black},
    stringstyle=\color{mylilas},
    commentstyle=\color{mygreen},
    showstringspaces=false,
    numbers=left,
    numberstyle={\tiny \color{black}},
    numbersep=9pt, 
    emph=[1]{for,end,break},emphstyle=[1]\color{blue},
    emph=[2]{word1,word2}, emphstyle=[2]{style},    
}

\begin{document}

\begin{titlepage}
    \begin{center}
        \Huge
        \vspace*{3cm}
        \textbf{ \quad Satellite Communication\\ \quad \quad Assignment 2} \\
        \qquad EEE F472\\[5ex]
        \quad\includegraphics[scale=0.75]{BITS_Pilani-Logo.png}
        \vfill
        \huge
        \qquad Sai Kartik\ \qquad \ \ \ (2020A3PS0435P)\\\qquad Rajeev Rajagopal (2020A3PS1237P) \\ \qquad Kshitij Merchant (2020A3PS0436P)
    \end{center}
\end{titlepage}

\pagenumbering{roman}

\tableofcontents
\listoftables
\newpage
\pagenumbering{arabic}
\setcounter{page}{1}

\section {Problem 1}

\begin{tcolorbox}
    \begin{itemize}
        \item Write the algorithm to compute position of the Friendship 7 spacecraft in the form of pseudocode (or present a flowchart)
        \item Write a computer program (in MATLAB) to compute position of the Friendship 7 spacecraft
    \end{itemize}

\end{tcolorbox}
\subsection{Pseudocode}
\begin{enumerate}
    \item Declare constants
    \item Input date, time (precisely)
    \item convert into fractional day with the value of time (fractional day = hh/24+mm/1440+ss/86400)
    \item if month $<$=2, year = year -1 ; month = month + 12
    \item Calculate Julian date with obtained params
    \item Calculate various orbital parameters
    \item Convert to cartesian co-ordinates from obtained angular orbital parameters. Alpha and Delta is obtained from here
    \item Compute GST with its empirical formula
    \item Make sure GST is in the range [0,360)
    \item Calculate L=GST - Alpha
    \item L = wrapTo180(L)
    \item Print Delta and L (lat. and long.)
\end{enumerate}

\newpage

\subsection{MATLAB Code}
\lstinputlisting{MATLAB codes/f7.m}
\newpage
\section {Problem 2}

\begin{tcolorbox}
    \begin{itemize}
        \item Consider a satellite non-linear power amplifier (e.g., TWTA) with the
              following voltage transfer characteristic:
              $$v_{out} = b_1v_{in}+b_3v_{in}^3$$
              Where $v_{in}(t) = \cos(2\pi f_1t)+\cos(2\pi f_2t)+\cos(2\pi f_3t)$. \\[1ex]
              Here, $f_1 = 3.8$GHz, $f_2 = 3.95$GHz, and $f_3 = 4.05$GHz.
        \item Assume that the satellite transponder bandwidth is from $f_{\ell} = 3.7$GHz to $f_u = 4.2$GHz
        \item Determine the amplitude and location of all intermodulation (IM) products within the transponder bandwidth resulting from the power amplifier action on the input signal $v_{in}$. \textbf{Note: }Present your final answers in the form of a table
    \end{itemize}

\end{tcolorbox}

\subsection{Solution}
Given
\begin{equation}
    v_{out} = b_1v_{in}+b_3v_{in}^3
\end{equation}
We have:
\begin{equation}
    v_{in} = \cos(\omega_1t)+\cos(\omega_2t)+\cos(\omega_3t) \ \ \text{where} \ \ \omega_1 = 2\pi f_1, \ \ \omega_2 = 2\pi f_2, \ \ \omega_3 = 2\pi f_3
\end{equation}
We know that:
\begin{equation}
    \label{eq:3}
    (a+b+c)^3 = a^3+3a^2b+3a^2c+6abc+3ac^2+b^3+3b^2c+3bc^2+c^3
\end{equation}
Using \ref{eq:3} we have $v_{in}^3$ to be:
\begin{align}
    \label{eq:4}
    \begin{split}
        v_{in}^3 & = \cos^3(\omega_1t)+3\cos^2(\omega_1t)\cos(\omega_2t)+3\cos^2(\omega_1t)\cos(\omega_3t)             \\
        & +6\cos(\omega_1t)\cos(\omega_2t)\cos(\omega_3t)+3\cos(\omega_1t)\cos^2(\omega_3t)+\cos^3(\omega_2t) \\
        & +3\cos^2(\omega_2t)\cos(\omega_3t)+3\cos(\omega_2t)\cos^2(\omega_3t)+\cos^3(\omega_3t)
    \end{split}
\end{align}
Applying the trigonometric identities given below:
\begin{align}
    \cos^3x & = \frac{\cos(3x)+3\cos x}{4} \\
    \cos^2x & = \frac{\cos(2x)+1}{2}
\end{align}
We can simplify \ref{eq:4} as follows
\begin{align}
    \begin{split}
        v_{in}^3 &= \frac{\cos(3\omega_1t)+3\cos(\omega_1t)}{4} + \frac{\cos(3\omega_2t)+3\cos(\omega_2t)}{4}+\frac{\cos(3\omega_3t)+3\cos(\omega_3t)}{4} \\
        & + \frac{3\cos(\omega_1t)\cos(\omega_2t)+3\cos(\omega_1t)\cos(\omega_3t)+3\cos(\omega_2t)\cos(\omega_3t)}{2} \\
        & + \frac{3\cos(\omega_1t)\cos(\omega_2t)\cos(\omega_3t)}{4}
    \end{split}
\end{align}
\newpage
Simplifying further, we arrive at:
\begin{align}
    \begin{split}
        v_{in}^3 &= \frac 14 \left(\cos(3\omega_1t)+\cos(3\omega_2t)+\cos(3\omega_3t)\right) \\
        & + \frac 34 \big[\cos\{(2\omega_1+\omega_2)t\}+\cos\{(2\omega_1-\omega_2)t\}\\&\ \quad + \cos\{(2\omega_1+\omega_3)t\}+ \cos\{(2\omega_1-\omega_3)t\}\\&\ \quad + \cos\{(2\omega_2+\omega_1)t\}+ \cos\{(2\omega_2-\omega_1)t\}\\&\ \quad + \cos\{(2\omega_2+\omega_3)t\}+ \cos\{(2\omega_2-\omega_3)t\}\\&\ \quad + \cos\{(2\omega_3+\omega_1)t\}+ \cos\{(2\omega_3-\omega_1)t\}\\&\ \quad + \cos\{(2\omega_3+\omega_2)t\}+ \cos\{(2\omega_3-\omega_2)t\}  \big] \\
        & + \frac 32 \big[\cos\{(\omega_1+\omega_2+\omega_3)t\}+\cos\{(\omega_1+\omega_2-\omega_3)t\}\\&\ \quad  +\cos\{(\omega_1-\omega_2+\omega_3)t\}+\cos\{(\omega_1-\omega_2-\omega_3)t\}\big] \\
        & + \frac {15}{4} \big[\cos(\omega_1t)+\cos(\omega_2t)+\cos(\omega_3t)\big]
    \end{split}
\end{align}
Hence the final value of $v_{IM}$ is (accounting only for frequencies in the range $f_{\ell} = 3.7$GHz to $f_u = 4.2$GHz):
\begin{align}
    \begin{split}
        v_{IM} &= \left(b_1+\frac{15}{4}b_3\right)\left(\cos(\omega_1t)+\cos(\omega_2t)+\cos(\omega_3t)\right) \\
        & + \frac 34 b_3 \left[\cos(\{2\omega_2-\omega_1\}t)+\cos(\{2\omega_2-\omega_3\}t)+\cos(\{2\omega_3-\omega_2\}t)\right] \\
        & + \frac 32 b_3 \left[\cos\{(\omega_1+\omega_2-\omega_3)t\}+\cos\{(\omega_1-\omega_2+\omega_3)t\}+\cos\{(\omega_1-\omega_2-\omega_3)t\}\right]
    \end{split}
\end{align}
Presenting the frequencies received in a tabular format:
\begin{center}
    \begin{table}[ht]
        \centering
        \begin{tabular}{|l|l|l|}
            \hline
            Amplitude                              & Location      & Frequency \\
            \hline
            \multirow{3}{*}{$b_1+\frac{15}{4}b_3$} & $f_1$         & 3.8GHz    \\\cline{2-3}
                                                   & $f_2$         & 3.95GHz   \\\cline{2-3}
                                                   & $f_3$         & 4.05GHz   \\
            \hline
            \multirow{3}{*}{$\frac 34 b_3$ }       & $2f_2-f_1$    & 4.1GHz    \\\cline{2-3}
                                                   & $2f_2-f_3$    & 3.85GHz   \\\cline{2-3}
                                                   & $2f_3-f_2$    & 4.15GHz   \\
            \hline
            \multirow{3}{*}{$\frac 32 b_3$ }       & $f_1+f_2-f_3$ & 3.7GHz    \\\cline{2-3}
                                                   & $f_1-f_2+f_3$ & 3.9GHz    \\\cline{2-3}
                                                   & $f_1-f_2-f_3$ & 4.2GHz    \\
            \hline
        \end{tabular}
        \caption{IM frequencies}
    \end{table}
\end{center}

\end{document}